\section{Tokenization}
\label{sec:tokenization}

Tokenization is implemented as multiu-step-process.
The individual steps are described in the following subsections.

\subsection{Normalization}
Initially, all characters are converted to lowercase and leading or trailing whitespace is removed from the input.
This ensures that the tokenization is case-insensitive.

\subsection{Word extraction}
The regular expression \verb=[.,()\[\]{}:;\n\t\s |/?!]= is used to split the normalized input at punctuation characters.
This approach leaves hyphenated words intact.
For example, the input \verb|non-numeric word| results in the output \verb|non-numeric| and \verb|word|.

\subsection{Word processing}
Each word created during the previous step is processed by performing the following actions.

\begin{enumerate}
  \item Split the word at the hyphen character \verb|-|.
  \item Create a list including each part created by splitting at hyphenation, as well as the initial word.
  \item For each item in the list, remove all non-word-characters using the regular expression \verb|[\Wˆ_]|.
\end{enumerate}

This method of handling hyphenation means that a hyphenated word results in multiple tokens.
For example, \verb|non-numeric| results in the output \verb|non|, \verb|numeric| and \verb|nonnumeric|.
In contrast to removal of hyphens in hyphenated words, this approach can handle input that substitutes hyphens with whitespace.

\subsection{World filtering}
The list of processed words is then flattened and filtered.
This filtering step removes all words that are either stopwords or non-numeric words with a length of less than 2.
A list of stopwords is obtained using the library \verb|nltk|.
Non-numeric words with a length of less than 2 are either stopwords ("I" or "a") or left-over artefacts from the word processing step.

\subsection{Stemming}
Lastly, the remaining words are stemmed as described in \cref{sec:stemming}.
The list of stemmed words is returned as the final tokens of an input string.
