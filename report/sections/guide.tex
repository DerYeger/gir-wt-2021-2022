\section{How to run your Prototype}
\label{sec:guide}

\subsection{Installation}

To run the program, Python 3.8+ and the following libraries are required.

\begin{itemize}
  \item \href{https://numpy.org}{NumPy}
  \item \href{https://github.com/CITGuru/PyInquirer}{PyInquirer}
  \item \href{https://github.com/prompt-toolkit/python-prompt-toolkit}{prompt\_toolkit}
  \item \href{https://www.crummy.com/software/BeautifulSoup/bs4/doc/}{BeautifulSoup}
  \item \href{https://www.nltk.org}{NLTK}
  \item \href{https://pypi.org/project/termcolor/}{termcolor}
\end{itemize}

Downloading the provided dataset and extracting its contents into the \verb|code/dataset| folder is neccessary.

\subsection{Usage}

Run \verb|python main.py| to start the program.
If no save data is found in the \verb|code/tables| directory, the program will index the dataset.
Otherwise, it will load the save data from the respective files.
Alternatively, running \verb|python main.py -c| or \verb|python main.py --clean| will always create a fresh index.
When a fresh index is being created, the program will ask how many files it is supposed to index.
To index all files, \verb|-1| must be entered.

\subsubsection{Evaluation mode}

Select \textit{Evaluation} in the menu to run the evaluation mode.
First, the program will parse the topics file \verb|code/dataset/topics.xml| and query the index for every topic twice using both *BM25* and *TFIDF* scoring methods.
Afterward, it will attempt to run trec\_eval for the evaluation results and save the scores in separate files.

% TODO Resolve duplication with previous description of evaluation mode.

\subsubsection{Exploration mode}

Selecting "Exploration" in the menu starts the exploration for exploration of the indexed documents with custom queries.
After choosing a scoring method and entering a query, the program will present a list of results in descending order.
Selecting an entry will print its contents.
Alternatively, choosing "Return" will exit the results-view.

\subsubsection{Other}


Selecting "Reset index\" will trigger the creation of a new index without restarting the program.

Selecting "Exit" will terminate the program.
