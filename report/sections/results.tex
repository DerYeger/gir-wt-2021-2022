\section{Evaluation Results}
\label{sec:results}

The following results were obtained on a 2020 M1 MacBook Pro with 16GB of RAM using Python 3.8.9 and trec\_eval 9.0.8 installed via Homebrew.

\medskip

Using the provided dataset, our program indexed 281782 articles of 553 files in 2294.86 seconds.

\medskip

In evaluation mode, our program parses the 107 topics of the provided dataset, once again using BeautifulSoup.
For each parsed topic, it then queries the inverted index using the topics's title twice.
Once using TFIDF scoring and a second time using BM25.
For querying, only the title of a topic is used.
The results of each query are stored in a file \verb|SCORING_evaluation.txt| in the \verb|retrieval_results| directory, where \verb|SCORING| is either \verb|tf-idf| or \verb|bm25|.

Afterward, the program checks if the command \verb|trec_eval| is available on the path.
If it is, the trec\_eval utility will be executed for each file mentioned above.
Its output will be written to \verb|SCORING_evaluation.txt.eval.txt| in the same directory.
Our evaluation results are pictured in \cref{table:results}.

Using Python's built-in \verb|time| package, we measured an evaluation time of 43122.22 milliseconds\footnote{Excluding the execution of trec\_eval.}.
Since two queries are executed for each of the 107 topics, a query took about 201.51 milliseconds on average.

\begin{table}[]
\center
\begin{tabular}{|l|l|l|l|l|}
	\hline
	      & mAP    & NDCG@10 & P@10   & R@10   \\
	\hline
	TFIDF & 0.1477 & 0.4967  & 0.4500 & 0.0971 \\
	\hline
	BM25  & 0.2637 & 0.6006  & 0.5596 & 0.1500 \\
	\hline
\end{tabular}
\caption{Evaluation Results}
\label{table:results}
\end{table}
