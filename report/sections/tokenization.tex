\section{Tokenization}
\label{sec:tokenization}

Our tokenization approach consists of the following steps in the order they are listed.

\paragraph{Normalization}
Initially, we convert all characters to lowercase and strip leading or trailing whitespace.
This ensures that the tokenization is case-insensitive.

\paragraph{Word extraction}
The regular expression \verb=[.,()\[\]{}:;\n\t\s |/?!]= is used to split the normalized word at punctuation characters.
This approach leaves hyphenated words intact.
For example, the input \verb|non-numeric word| results in the output \verb|non-numeric| and \verb|word|.

\paragraph{Word processing}
For each word created during the previous step, we perform the following actions.

\begin{enumerate}
  \item Split the word at the hyphen character \verb|-|.
  \item Create a list including each part created by splitting at hyphenation, as well as the initial word.
  \item For each item in the list, remove all non-word-characters using the regular expression \verb|[\Wˆ_]|.
\end{enumerate}

This method of handling hyphenation means that an input \verb|non-numeric| results in the output \verb|non|, \verb|numeric| and \verb|nonnumeric|.
In contrast to a simple removal of hyphens in hyphenated words, this approach can also handle input that substitutes hyphens with whitespace.

\paragraph{Word filtering}
We then continue by filtering the flattened list of lists obtained by executing the previous step.
This filtering step removes all words that are either stopwords or non-numeric words with a length of less than 2.
A list of stopwords is obtained using the library \verb|nltk|.
Non-numeric words with a length of less than 2 are either stopwords ("I" or "a") or left-over artefacts from the word processing step.

\paragraph{Stemming}
Lastly, we stem the remaining words as described in \cref{sec:stemming}.
The list of stemmed words is returned as the final tokens of an input string.




% TODO Tokenization: Describe method
