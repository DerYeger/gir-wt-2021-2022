\section{Training new language models}

\autocite{dataset} For training a German language model, a dataset of three million sentences from newspapers by the Leipzig Corpora Collection is used\footnote{Obtained from \url{https://www.kaggle.com/rtatman/3-million-german-sentences/version/1?select=deu_news_2015_3M-sentences.txt} on 27th December 2021.}.
For version control compatibility, it was split into multiple parts using the Linux command \verb|split -l 100000| \verb|--additional-suffix=.txt| \verb|$FileName dataset.txt|.
Before tokenization, the dataset measures a size of 0.32GB.
This falls in the suggested range mentioned in the assignment.

The tab-separated number at the start of each line is removed by the processing.
To match the model, queried words were processed with the same tokenization as the dataset.
It should be noted that the results can vary, because the untrained model is initialized with random weights.

\begin{table}[hb]
\center
\begin{tabular}{|l|l|l|l|}
\hline
\textbf{Word} & \textbf{Top-1}     & \textbf{Top-2} & \textbf{Top-3}     \\ \hline
deutschland   & österreich         & europa         & frankreich         \\ \hline
politik       & wirtschaftspolitik & außenpolitik   & flüchtlingspolitik \\ \hline
kanzler       & bundeskanzler      & sozialdemokrat & parteichef         \\ \hline
\end{tabular}
\end{table}

The results indicate that the model training was successful.
For all three words, the top three most similar words are semantically related.

\paragraph{deutschland}
The connections to \enquote{österreich} and \enquote{frankreich} are trivial, as both are neighbouring countries that share a long history with Germany.
Since Germany is located in central Europe, it also makes sense to see \enquote{europa} in the top three most similar words.
The order of the words is also reasonable, as Austria is much more similar to Germany than France in many aspects like language and culture.

\paragraph{politik}
All three words contain the word \enquote{politik} itself and are semantically related to it.
They are commonly the subjects of newspaper articles.

\paragraph{kanzler}
Since \enquote{kanzler} is the short form of \enquote{bundeskanzler}, it makes sense to see the latter as the most similar word.
\enquote{sozialdemokrat} being the second most similar word is sound, as the last male chancellor was a social democrat and the social democrat had the chancellorship multiple times.
Lastly, \enquote{parteichef} is related by the CDU party having the tradition of its party leader also being their candidate for chancellorship.
